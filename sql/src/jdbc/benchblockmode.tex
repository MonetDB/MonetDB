\documentclass{article}

\begin{document}

\section{Introduction}

The MonetDB Mapi protocol knows two modes of communication;
(1) line mode, (2) block mode. The JDBC driver for MonetDB can
use both modes when communicating to Monet, having an almost
transparent result on its behavior.


\subsection{Line mode protocol}

The line mode protocol of MonetDB is the simplest communication
method, and very suitable for understanding the dialogs between
the server and the driver.

In line mode, each newline character is a separator, indicating
something is finished for processing. In its simple form, line mode
makes it possible to read line by line from the communication socket
and write line by line back to the server.


\subsection{Block mode protocol}

The more advanced block mode protocol uses `blocks' of data which
are sent at once. Separators in this mode are `zero terminated' blocks
which indicate that everything was sent.

Block mode requires both sender and receiver to tell what size the
block to be sent is and terminate it with zero. Since this size is sent
over using an integer, both sender and receiver should be aware of
their endianness\footnote{Computer systems are either little or big
endian. The difference is in whether the least or most significant byte
is sent first.} before they can understand the lengths sent over.


\section{Implications for Java}

Using line or block mode has effects on how the JDBC driver has to
handle sending and receiving data. Since Java always runs in a
virtual machine, there is for programs no direct access to the
operating system and a uniform machine on each platform is created.
This is especially with block mode an issue.


\subsection{Line mode}

In line mode, the JDBC driver simply uses \textsf{Reader}s and
\textsf{Writer}s for its communication with the server. \textsf{Reader}s
and \textsf{Writer}s are character oriented communication sockets,
and support a \textsf{readLine} method, resulting in a very
simple implementation when dealing with line mode.


\subsection{Block mode}

When using block mode, the JDBC driver has to use \textsf{Stream}s
for communicating with the server, for there is not only character data
to be read and sent.

The Java virtual machine emulates a big endian machine. On a little
endian host, however, this will result in lots of swapping of bytes.
In the implementation of the JDBC driver it was chosen to use newly
added features of Java 1.4 in order to be able to have a byte array
using the host specific endianness. Using a native byte array, we
expect --- especially if the server resides on a platform with the same
endianness --- a higher performance.

For each block the JDBC driver sends or receives the length has
to be calculated or converted. Since Java itself is big endian, there
always occurs some byte swapping, or at least this has to be checked
before processing. This adds some extra overhead.

Since blocks are sent, and results come separated by newlines,
received blocks need to be concatenated, and newlines need to be
searched for in the blocks. This results in some buffering and
searching for newline characters, which was done automatically
in line mode by the \textsf{Reader}'s \textsf{readLine} method.


\section{Differences for the JDBC user}

Although the JDBC driver tries to make it transparent to the user, which
mode is used, there are some differences. There is one difference in
behavior and some differences in performance.


\subsection{Behavioral difference}

When using line mode, queries that are sent cannot contain newlines.
This restriction can be related directly to the line mode protocol, since
newlines are treated as separators. As a result of this, a user cannot
send a multi-line query. Block mode however, since it does not treat
the newline as separator, can handle such queries. Since many
query applications have graphical enhancements which encourage
the use of newlines in queries, block mode is an enabler for those
applications.


\subsection{Performance differences}

Since block mode requires more control commands in order to send and
receive a query, requires byte conversion and has the need for a
FIFO\footnote{First In First Out} buffer it is considered a heavier
communication method than the line mode variant.

Block mode, however is more sophisticated and is said to be less
intensive for the server.

We benchmarked the JDBC driver using block mode and line mode on
the VOC database. We dumped the database and restored it.

The system we used was a AMD Athlon XP 2500+, 512MB memory
and a Maxtor 80 GB hard disk with 8MB cache. The running operating
system was Gentoo Linux 2.6.5. The system was just running the
Gnome display manager and normal daemons. We did not try to
reduce influence from other processes as much as possible, we just
closed all running applications and made sure nothing was started
in between.

In order to dump and restore the database using the JDBC driver, the
\textsf{JdbcClient} utility was used. For dumping the\textsf{-D} flag
was used while restoring was done by feeding the dumped file to the
client using the \textsf{-f} flag.

For each benchmark we (re)started Mserver and ran once the to be
run test to let the system initialize and cache whatever it liked.
After this, initial run, we started three tests in sequence.


\subsubsection{Queries with large result sets}

The VOC database consists of a few tables around 8000 rows each.
Dumping this database in general means reading large result sets
returned by the server. In table \ref{tab:dump} the results of our
benchmark can be found.

\begin{table}
\caption{Dump performance results using the VOC database}
\centering
\begin{tabular}{| l | r r r || r |}
\hline
Mode & \multicolumn{3}{l ||}{Time} & \multicolumn{1}{l |}{Average} \\
\hline\hline
Block mode & 0m8.470s & 0m8.804s & 0m8.605s & 0m8.626s \\
Line mode & 0m20.029s & 0m19.632s & 0m19.488s & 0m19.716s \\
\hline
\end{tabular}
\label{tab:dump}
\end{table}

It might be needless to say that block mode approximately performs
twice as fast as line mode, for reading data.


\subsubsection{Large amounts of resultless queries}

Restoring the VOC database basically means inserting all the rows
dumped, back into the database. \texttt{INSERT INTO} statements
have little or no result output, causing in general a lot of control
commands overhead using block mode. The efficiency of line mode
is considered to be higher. Table \ref{tab:restore} shows the results
of restoring the VOC database.

\begin{table}
\caption{Restore performance results using the VOC database}
\centering
\begin{tabular}{| l | r r r || r |}
\hline
Mode & \multicolumn{3}{l ||}{Time} & \multicolumn{1}{l |}{Average} \\
\hline\hline
Block mode & 1m36.517s & 1m36.639s & 1m36.536s & 1m36.564s \\
Line mode & 1m33.233s & 1m33.821s & 1m35.651s & 1m34.235s \\
\hline
\end{tabular}
\label{tab:restore}
\end{table}

The average difference between line and block mode is 2 seconds in
favor of line mode, which is considered to be a very small difference
regarding the amount of data.


\subsubsection{Conclusions}

Although it was expected that block mode would perform slower, it
seems to outperform line mode quite easily. Most probably do the ---
for the driver --- expensive investments on the communication
protocol, result in a much more efficient situation for the server, which
in total performs better.

Remarkable is the two times improvement on large result queries, while
the difference on resultless queries is very little. Added that multi line
queries are allowed we came to the conclusion that the JDBC driver
should by default use block mode instead of line mode.


\section{Discussion}

In some situations it might be useful to switch the default mode to
another mode. For instance, when it's known that there will only be
sent updates or inserts over the connection, it might be useful to
use line mode as clear from table \ref{tab:restore}.

The driver allows the user to switch the mode it uses for connecting.
When the mode is set before a \textsf{Connection} is retrieved via
for instance the \textsf{DriverManager} the newly requested
\textsf{Connection} will use the given mode.

An example of code that would request a \textsf{Connection} in line
mode:
\begin{verbatim}
Class.forName("nl.cwi.monetdb.jdbc.MonetDriver");
Connection con = DriverManager.getConnection(
 	"jdbc:monetdb://localhost/test?blockmode=false", "test", "test");
\end{verbatim}

\end{document}
